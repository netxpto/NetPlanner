\clearpage

\section{Local Oscillator}

This block simulates a local oscillator which can have shot noise or not. It produces one output complex signal and it doesn't accept input signals.

\subsection*{Input Parameters}

\begin{itemize}
	\item opticalPower\{ 1e-3 \}
	\item wavelength\{ 1550e-9 \}
	\item frequency\{ SPEED\_OF\_LIGHT / wavelength \}
	\item phase\{ 0 \}
	\item samplingPeriod\{ 0.0 \}
	\item shotNoise\{ false \}
\end{itemize}

\subsection*{Methods}

LocalOscillator() {}
\bigbreak
LocalOscillator(vector$<$Signal *$>$ \&InputSig, vector$<$Signal *$>$ \&OutputSig) :Block(InputSig, OutputSig)\{\};
\bigbreak
void initialize(void);
\bigbreak
bool runBlock(void);
\bigbreak
void setSamplingPeriod(double sPeriod);
\bigbreak
void setOpticalPower(double oPower);
\bigbreak
void setOpticalPower\_dBm(double oPower\_dBm);
\bigbreak
void setWavelength(double wlength);
\bigbreak
void setPhase(double lOscillatorPhase);
\bigbreak
void setShotNoise(bool sNoise);

\subsection*{Functional description}

This block generates a complex signal with a specified phase given by the input parameter \textit{phase}.

It can have shot noise or not which corresponds to setting the \textit{shotNoise} parameter to True or False, respectively. If there isn't shot noise the the output of this block is given by $0.5*\sqrt{OpticalPower}*ComplexSignal$. If there's shot noise then a random gaussian distributed noise component is added to the \textit{OpticalPower}.

\pagebreak
\subsection*{Input Signals}

\subparagraph*{Number:} 0

\subsection*{Output Signals}

\subparagraph*{Number:} 1

\subparagraph*{Type:} Optical signal

\subsection*{Examples}

\subsection*{Sugestions for future improvement}


