\documentclass[5pt]{article}
\usepackage{mathptmx,amsmath}
\usepackage{pdfslide2,pause}
\usepackage{eurosym}
\usepackage[portuguese,english]{babel}
\usepackage{kerkis}
\usepackage{colortbl} % used to highlight row or columns of tables. http://www.tug.org/pracjourn/2007-1/mori/mori.pdf
\usepackage[small]{caption} % more option on http://www.dd.chalmers.se/latex/Docs/PDF/caption.pdf
\usepackage[tight,scriptsize]{subfigure}
\usepackage{lastpage}
\usepackage{chngcntr}
\usepackage[absolute,overlay]{textpos}
\usepackage{tabto}
\usepackage{animate}
%\usepackage{listings}
\captionsetup{labelformat=empty,skip=-0.8cm}

%\lstset{
%    language=Matlab,                % choose the language of the code
%    basicstyle=\ttfamily\tiny,       % the size of the fonts that are used for the code
%    numbers=none,                   % where to put the line-numbers
%    numberstyle=\tiny,              % the size of the fonts that are used for the line-numbers
%    stepnumber=1,                   % the step between two line-numbers. If it's 1 each line will be numbered
%    numbersep=5pt,                  % how far the line-numbers are from the code
%    backgroundcolor=\color{white},  % choose the background color. You must add \usepackage{color}
%    showspaces=false,               % show spaces adding particular underscores
%    showstringspaces=false,         % underline spaces within strings
%    showtabs=false,                 % show tabs within strings adding particular underscores
%    tab=\rightarrowfill,
%    frame=none,	                    % adds a frame around the code
%    tabsize=2,	                    % sets default tabsize to 2 spaces
%    captionpos=b,                   % sets the caption-position to bottom
%    breaklines=true,                % sets automatic line breaking
%    breakatwhitespace=false,        % sets if automatic breaks should only happen at whitespace
%    title=\lstname,                 % show the filename of files included with \lstinputlisting; also try caption instead of title
%    escapeinside={\%*}{*)},          % if you want to add a comment within your code
%    morekeywords={ifftshift,fftshift},
%    keywordstyle=\bfseries\color[rgb]{0,0,0.3},
%    commentstyle=\color[rgb]{0.133,0.5,0.133}
%}
%\lstset{
%    emph={function,end,for,if,while},
%    emphstyle=\bfseries\color[rgb]{0.6,0,0},
%}

\definecolor{itblue}{rgb}{0.0,0.0,0.5}
\definecolor{itred}{rgb}{0.82,0.18,0.24}
\newcommand{\pageNum}{
    \begin{picture}(0,0)(0,0)
        \put(-15,-390){
            \begin{minipage}{1.8cm}
            \end{minipage}
        }
    \end{picture}
}
\newcommand{\cb}[1]{{\color{itblue} #1}}%
\newcommand{\cred}[1]{{\color{itred} #1}}%
\newcommand{\bb}[1]{{\textbf{\color{itblue} #1}}}%
\newcommand{\br}[1]{{\textbf{\color{itred} #1}}}%
\renewcommand{\labelitemi}{\textcolor{itred}{\normalsize $\bullet$}}
\renewcommand{\labelitemii}{\textcolor{itblue}{$\bullet$}}
\newcommand{\mysection}[1]{\section*{\pageNum\color{itred}\sffamily #1}\vspace*{0.5cm}\overlay{it_1.png}\sffamily}%
\newcommand{\ITfootnote}[1]{\hspace{1.8cm}\begin{minipage}{13cm}\tiny{#1}\end{minipage}}
\newcommand{\edfaGain}{$G=\exp\left(\frac{\alpha}{2}L_{span}\right)$}
\newenvironment{reference}{
    \begin{textblock*}{0.7\textwidth}(32mm,137mm)\tiny\noindent\bgroup\color{black}
}
{
    \egroup\end{textblock*}
}


\graphicspath{{./Figures/}}
\pagestyle{title}

\hyphenpenalty=50000
\tolerance=10000

\setlength{\textheight}{1.5\textheight}

%%%%%%%%%%%%%%%%%%%%%%%%%%%%%%%%%%%%%%%%%%%%%%%%%%%%%%%%%%%%%%%%%%%%%%%%%%%%%%%%%%%%%%%%%%%%%%%%%%%
%%%%%%%%%%%%%%%%%%%%%%%%%%%%%%%%%%%%%%%%%%%%%%%%%%%%%%%%%%%%%%%%%%%%%%%%%%%%%%%%%%%%%%%%%%%%%%%%%%%

\begin{document}

%************************************************************************************************
%                                          SLIDE
%************************************************************************************************
\pagenumbering{roman}
\begin{titlepage}  \overlay{it_0.png}

\color{itblue} \sffamily \noindent \small
\hspace*{1cm}  Universidade de Aveiro\\ %Instituto\\ Superior T�cnico, Instituto de Telecomunica��es\\
\hspace*{1cm}  2016-2017\\ %Lisboa, 14th of February, 2013\\

\vspace*{1cm}
\begin{center}
    \color{black} \sffamily \noindent \Large
    \br{Oblivious Transfer Protocol\\}
\end{center}
\vspace{6mm}
\begin{center}
    \color{black}
    \textbf{Mariana Ferreira Ramos\\}
    {(marianaferreiraramos@ua.pt)}
\end{center}

\vspace{0.0mm}
\scriptsize
\begin{center}
Department of Electronics, Telecommunications and Informatics,\\
University of Aveiro, Aveiro, Portugal\\
Instituto de Telecomunica\c{c}\~{o}es, Aveiro, Portugal\\
\end{center}

\vspace{1.0cm}
\hspace*{13.2cm}\tiny \copyright 2005, it - instituto de telecomunica\c{c}\~{o}es\hfill

\end{titlepage}


\renewcommand{\headsep}{-25pt}
\pagenumbering{arabic}

%--------------------------------------------------------------------------------------------------
%------------ SLIDE -------------------------------------------------------------------------------

\mysection{ \\[-5mm]1-out-of-2 OT Protocol: starting conditions}\large
\vspace*{0mm}
\begin{itemize}
    \item {Alice has two messages $m_1$ and $m_2$ and Bob wants to know one of them, $m_b$, without Alice knowing which one, i.e. without Alice knowing $b$, and Alice wants to keep the other message private, i.e. without Bob knowing $m_{\bar{b}}$.}\\[-5mm]
    \item {First of all, Alice and Bob must know two parameters: message length $s$ and the expansion factor $k$.}
    \item {Two basis are required: $'+'$ rectilinear basis and $'\times'$ diagonal basis.}
    \item {For rectilinear basis we defined as a binary 0 the polarization of $0^{\circ}$ and a binary 1 the polarization of $90^{\circ}$.}
    \item {For diagonal basis we defined as a binary 0 the polarization of $-45^{\circ}$ and a binary 1 the polarization of $45^{\circ}$. }
\end{itemize}



%--------------------------------------------------------------------------------------------------
%------------ SLIDE -------------------------------------------------------------------------------

\mysection{ \\[-5mm]1-out-of-2 OT Protocol: starting conditions}\large
\vspace{5mm}
\begin{center}
  \begin{tabular}{|c|c|c|c|}
  \hline
  % after \\: \hline or \cline{col1-col2} \cline{col3-col4} ...
  \textbf{Bit} & \textbf{Basis} & \textbf{Degrees} & \textbf{Polarization} \\
  0 & $+$ & $0^{\circ}$ & $\longrightarrow$ \\
  0 & $\times$ & $-45^{\circ}$ & $\searrow$ \\
  1 & $+$ & $90^{\circ}$ & $\uparrow$ \\
  1 & $\times$ & $45^{\circ}$ & $\nearrow$ \\
  \hline
\end{tabular}

\end{center}
\vspace{5mm}
\begin{itemize}
  \item Alice has two messages to send to Bob: $m_{0} = \{0 0 1 1\}$ and $m_{1} = \{0 0 0 1\}$.
  \item Lets assume that in this example Alice and Bob knows two start parameters: the message's size $s=4$ and a expansion factor $k=2$.
\end{itemize}

%--------------------------------------------------------------------------------------------------
%------------ SLIDE -------------------------------------------------------------------------------

\mysection{ \\[-5mm]1-out-of-2 OT Protocol: Description}\large
\vspace*{0mm}
\begin{description}
  \item[Step 1] Alice randomly generates two bit sequences, with $ks$ length:
  \begin{center}
    \begin{tabular}{c|ccccccccc}
      $S_{A1}$ & 0 & 1 & 1 & 0 & 0 & 1 & 0 & 1 & \textbf{Basis}\\
      & $+$ & $\times$ & $\times$ & $+$ & $+$ & $\times$ & $+$ & $\times$ &\\ \hline
    \end{tabular}
    
    \vspace{5mm}
    
    \begin{tabular}{c|ccccccccc}
      $S_{A2}$ & 1 & 1 & 0 & 0 & 0 & 1 & 0 & 0 & \textbf{Keys}\\
      & $\uparrow$ & $\nearrow$ & $\searrow$ & $\longrightarrow$ & $\longrightarrow$ & $\nearrow$ & $\longrightarrow$ & $\searrow$ &\\ \hline
    \end{tabular}
  \end{center}
  
  \vspace{3mm}
  
  \item[Step 2] Alice sends to Bob throughout a quantum channel $ks$ photons encrypted using the basis defined in $S_{A1}$ and according to the keys defined in $S_{A2}$.
      \begin{center}
        $S_{AB} = \{\uparrow, \nearrow, \searrow, \to, \to, \nearrow, \to, \searrow \}$
      \end{center}
  
\end{description}

%--------------------------------------------------------------------------------------------------
%------------ SLIDE -------------------------------------------------------------------------------

\mysection{ \\[-5mm]1-out-of-2 OT Protocol: Description}\large
\vspace*{0mm}

\begin{description}
  \item[Step 3] Bob also randomly generates $ks$ bits. Lets assume:
    \begin{center}
        $S_{B1} = \{0,1,0,1,0,1,1,1 \}$.
    \end{center}
    
     When Bob receives photons from Alice, he measures them using the basis defined in $S_{B1}$:
     
     \begin{center}
       $\{ +,\times,+,\times,+,\times, \times, \times \}$
     \end{center}
     
      Bob will get $ks$ results:
      
      \begin{center}
        $S_{B1'} = \{1,1,0,1,0,1,1,0 \}$
      \end{center}
  
  \item[Step 4] Bob sends to Alice an Hash Function value HASH1, which will do Bob's commitment with the measurements done.
\end{description}

%--------------------------------------------------------------------------------------------------
%------------ SLIDE -------------------------------------------------------------------------------

\mysection{ \\[-5mm]1-out-of-2 OT Protocol: Description}\large
\vspace*{0mm}

\begin{description}
  \item[Step 5] When Alice receives HASH1, she sends throughout a classical channel the basis she used to encode the photons. In this case, we have assumed:
      
      \begin{center}
        $S_{A1} = \{0,1,1,0,0,1,0,1\}$
      \end{center}
      
  \item[Step 6] In order to know if he measured the photons correctly, Bob does the operation $S_{B2}=S_{B1} \oplus S_{A1}$. 
      
      \begin{center}
        \begin{tabular}{c|c c c c c c c c}
         $S_{B1}$ & 0 & 1 & 0 & 1 & 0 & 1 & 1 & 1 \\
         $S_{A1}$ & 0 & 1 & 1 & 0 & 0 & 1 & 0 & 1 \\ \hline
         $\oplus$ & 1 & 1 & 0 & 0 & 1 & 1 & 0 & 1
        \end{tabular}
      \end{center}
      
      The values $'1'$ correspond to the values he measured correctly and $'0'$ to the values he just guessed. Thus, $S_{B2} = \{1,1,0,0,1,1,0,1 \}$.

\end{description}
%--------------------------------------------------------------------------------------------------
%------------ SLIDE -------------------------------------------------------------------------------

\mysection{ \\[-5mm]1-out-of-2 OT Protocol: Description}\large
\vspace*{0mm}
 
 \begin{description}
   \item[Step 6 (cont)] Bob sends to Alice, through a classical channel, $n=min(\#0,\#1)=3$, where $\#0$ represents the number of zeros in $S_{B2}$ and $\#1$ the number of ones in $S_{B2}$.
   \item[Step 7] If $n<s$, Alice and Bob will repeat the steps from $1$ to $7$. In this case, $n=3$ which is smaller than $s=4$. Therefore, Alice and Bob repeat the steps from 1 to 7 in order to enlarge Bob's sets $S_{B1}$ and $S_{B2}$ as well as Alice's sets $S_{A1}$ and $S_{A2}$.
   \item[Step 8] Lets assume :
        \begin{center}
          $S_{B1}= \{1,1,0,0,0,1,0,0,1,0,0,0,0,0,1,1 \},$
          $S_{A1}=\{0,1,1,0,0,1,0,1,1,1,0,0,1,1,1,0 \},$
          $S_{A2}=\{1,1,0,0,0,1,0,0,1,0,1,0,0,0,1,1 \}.$
        \end{center}
        
       
 \end{description}

%--------------------------------------------------------------------------------------------------
%------------ SLIDE -------------------------------------------------------------------------------

\mysection{ \\[-5mm]1-out-of-2 OT Protocol: Description}\large
\vspace*{0mm}

\begin{description}
  \item[Step 8 (cont)]  Finally, for $S_{B2}=S_{B1} \oplus S_{A1}$:
        \begin{center}
          $S_{B2}= \{1,1,0,0,1,1,0,1,0,1,0,0,1,1,0,1 \}.$
        \end{center}

        Note that the sets were enlarge in the second iteration.
  \vspace{5mm}
  \item[Step 9] At this time, Bob sends again to Alice, through a classical channel, $n=min(\#0,\#1)=7$.
  \vspace{5mm}
  \item[Step 10] Alice checks if $n>s$ and acknowledge to Bob that she already knows that $n>s$. In this case, $n=7$ and $s=4$ being $n>s$ a valid condition.
\end{description}




%--------------------------------------------------------------------------------------------------
%------------ SLIDE-------
\mysection{1-out-of-2 OT Protocol: Description}\large
\vspace{0.4cm}
\begin{description}
  \item[Step 11] Bob defines two new sub-sets, $I_{0}$ and $I_{1}$.In this example, Bob defines two sub-sets with size $s=4$:
      \begin{center}
        $I_{0}=\{3,4,7,11 \},$
        $I_{1}= \{2,5,6,13 \}.$
      \end{center}
      Bob sends to Alice the set $S_{b}$.
      If Bob wants to know $m_{0}$ he must send to Alice throughout a classical channel the set $S_{0}=\{I_{1},I_{0} \}$, otherwise if he wants to know $m_{1}$ he must send to Alice throughout a classical channel the set $S_{1}=\{I_{0},I_{1} \}$.
  \item[Step 12]  With both the received set $S_{b}$ and the hash function value HASH1, Alice must be able to prove that Bob has being honest. \textbf{HOW???}
\end{description}

%--------------------------------------------------------------------------------------------------
%------------ SLIDE-------
\mysection{1-out-of-2 OT Protocol: Description}\large
\vspace{0cm}

\begin{description}
  \item[Step 13] Lets assume Bob sent $S_{0}=\{I_{1},I_{0} \}$. Alice defines two encryption keys $K_{0}$ and $K_{1}$ using the values in positions defined by Bob in the set sent by him. In this example, lets assume: 
      \begin{center}
        $K_{0}=\{1,0,1,0\}$ and $K_{1}=\{0,0,0,1\}.$
      \end{center}
  
       Alice does the operation  $m = \{m_{0}\oplus K_{0}, m_{1} \oplus K_{1} \}.$
       \vspace{5mm}
       \begin{center}
         \begin{tabular}{c|c c c c c c c c}
             $m_{0}$ & 0 & 0 & 1 & 1 \\
             $K_{0}$ & 1 & 0 & 1 & 0 \\ \hline
             $\oplus$ & 1 & 0 & 0 & 1
         \end{tabular}
         \quad
         \begin{tabular}{c|c c c c c c c c}
             $m_{1}$ & 0 & 0 & 0 & 1 \\
             $K_{1}$ & 0 & 0 & 0 & 1 \\ \hline
             $\oplus$ & 0 & 0 & 0 & 0
         \end{tabular}
       \end{center}
       \vspace{5mm}  
       
    Adding the two results, Alice will send to Bob the encoded message $m=\{1,0,0,1,0,0,0,0\}.$
\end{description}


%--------------------------------------------------------------------------------------------------
%------------ SLIDE--------------------------------------------------------------------------------
\mysection{1-out-of-2 OT Protocol: Description}\large
\vspace{0cm}

\begin{description}
  \item[Step 14] When Bob receives the message $m$, in the same way as Alice, Bob uses $S_{B1'}$ values of positions given by $I_{1}$ and $I_{0}$ and does the decrypted operation:
      
      \begin{center}
        \begin{tabular}{c|c c c c c c c c}
         $m$ & 1 & 0 & 0 & 1 & 0 & 0 & 0 & 0 \\
             & 1 & 0 & 1 & 0 & 0 & 1 & 1 & 0 \\ \hline
         $\oplus$ & 0 & 0 & 1 & 1 & 0 & 1 & 1 & 0 \\
        \end{tabular}
      \end{center}
      
      \vspace{6mm}
      The first four bits corresponds to message 1 and he received $\{0,0,1,1\}$, which is the right message $m_{0}$ and $\{0,1,1,0\}$ which is a wrong message for $m_{1}$.
      

\end{description}

%--------------------------------------------------------------------------------------------------
%------------ SLIDE--------------------------------------------------------------------------------
\mysection{1-out-of-2 OT Protocol: Open Issues}\large
\vspace{0.4cm}
\textbf{Steps $4$ and $12$ are not fully defined.}
\begin{enumerate}
  \item In step $4$ Bob may says to Alice that he has already measured the photon and it could be a lie. In order to prevent this an Hash Function must be used.

  \item In step $12$ Bob may uses some values in a dishonest way, i.e Bob can pick values from $I_{1}$ which he knows they are correct and send them in $I_{0}$ in order to know correct information about message $m_{\bar{b}}$.
\end{enumerate}

This problems can hopefully be solved using \textit{Bit Commitment} through \textit{Hash Functions}.






%-------------------------------------------------------------------
%------------ SLIDE ------------------------------------------------
\mysection{} \sffamily
\vspace{-10mm}
\large\centerline{E-mail: marianaferreiraramos@ua.pt}


\end{document}
